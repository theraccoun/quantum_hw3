\documentclass[12pt]{article}
\usepackage{amsmath}
\newcommand{\piRsquare}{\pi r^2}

\title{Quant. Comp. HW - 2}
\author{Steven MacCoun}
\date{Oct. 18, 2005}

\begin{document}
\maketitle						% automatic title!



\section{Simon's Problem}

(a) There are 128 possible outputs
\\(b) See attached source code: \\00110011000011101001100000111111110111101100000010000001001110011010100001
110001001000001110111000101001001011001010010010010001
\[
\boxed{67866407927622586744034130520498283665}
\]
\\(c) I reported an average of 152 trials
\\(d) 

\section{Modular Exponentiation}

Here was my python code:

\begin{verbatim}
def successive_squaring(base, exponent, modulus):
         e = 1
         vals = {}
         while e <= exponent:
                   sq = str(pow(base, e, modulus))
                  vals[e] = sq
                  print base , "^" , e, "    " , sq
                 e *= 2

         ex = exponent
         keys = sorted(vals)
         print keys
	
         i = len(keys)-1
         total = 1
         while ex > 1:
                  k = keys[i]
                  i -= 1
                  if ex - k < 0:
                           continue
                  else:
                            total = (total*int(vals[k])) % modulus
                            ex -= k

          print total

successive_squaring(1234, 1234*1234, int(math.pow(10,10)))
\end{verbatim}
And the output was: \[\boxed{3102217216.0}\] 


\section{RSA Misuse}

I first tried to solve this as strictly a math problem, but had little success, in large part because I thought that
the gcd(e1, e2) was somehow irrelevant to the problem. However, I noticed that normally the exponents are the same value
when performing RSA, so I scoured google to see if there was some well known attack where you have a common modulus with
different attacks. Turns out that it is fairly well documented, and Simmons wrote a paper on it a while back.

The basic idea is:
Since \[gcd(e_1,e_2) = 1\],
then \[\exists u, v   \;  s.t.   \;  e_1*u + e_2*v = 1\]
To solve for u and v, I used the extended Euclidean algorithm.
I then raise each side to u and v
\[c1^u = (M^{e1})^{u} mod \; n\]
\[c2^v = (M^{e2})^{v} mod \; n\]
\[c1^u * c2^v =  (M^{e1})^{u}*(M^{e2})^{v} mod \; n =  M^{e1*u + e2*v} mod \; n = M mod \; n\]

From this I can multiply $c1^d$ and $(c2^f)^{-1}$: \[c1^d * (c2^f)^{-1} = M^{bd}M^{-ce} = M^{bd-ce}=M mod  \; n\]
Because my modular exponentiation code can't handle negatives, note that $c2^{-v} \; mod \; n = c2^{n-f} mod \; n$


\section{Prime factorization}

Problem: Consider n=121932632103337941464563328643500519
\\\\(a) How many bits is n?
\begin{verbatim}
print len(str(121932632103337941464563328643500519))
\end{verbatim}
Output:  \[\boxed{36}\] 
\\(b) Find if n is prime with program that runs in less than one second.
\begin{verbatim}
def miller_rabin_pass(a, s, d, n):
         a_to_power = pow(a, d, n)
         if a_to_power == 1:
                   return True
         for i in xrange(s-1):
                if a_to_power == n - 1:
                          return True
                a_to_power = (a_to_power * a_to_power) % n
        return a_to_power == n - 1

def miller_rabin(n):
        #compute s and d
         d = n - 1
         s = 0
        while d % 2 == 0:
                  d >>= 1
                  s += 1

         #Run several miller_rabin passes
         for repeat in xrange(20):
                   a = randint(2, n-1)
                   if not miller_rabin_pass(a, s, d, n):
                            return False
         return True

print miller_rabin(n)

\end{verbatim}

\[
\boxed{False}
\]

(c) Simple trial divisions only need to try up to  $sqrt{n}$ in the worst case. So \[
\boxed{O(2^{n/2})}
\]

(d) 

(e) See attached code



\end{document} 


