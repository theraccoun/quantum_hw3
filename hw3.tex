\documentclass[12pt]{article}
\usepackage{amsmath}
\newcommand{\piRsquare}{\pi r^2}

\title{Quant. Comp. HW - 2}
\author{Steven MacCoun}
\date{Oct. 18, 2005}

\begin{document}
\maketitle						% automatic title!



\section{Simon's Problem}


\section{Modular Exponentiation}

Here was my python code:

\begin{verbatim}
#Modular Exponentiation
import math

def modular_exponentiation(base, exponent, modulus):
         c = 1
         for e_prime in range(1, exponent+1):
                    c = (c * base) % modulus
        return c

print modular_exponentiation(1234, 1234*1234, math.pow(10, 10))
\end{verbatim}
And the output was: \[\boxed{3102217216.0}\] 


\section{RSA Misuse}

\section{Prime factorization}

Problem: Consider n=121932632103337941464563328643500519
\\\\(a) How many bits is n?
\begin{verbatim}
print len(str(121932632103337941464563328643500519))
\end{verbatim}
Output:  \[\boxed{36}\] 
\\(b) Find if n is prime with program that runs in less than one second.
\begin{verbatim}
def miller_rabin_pass(a, s, d, n):
         a_to_power = pow(a, d, n)
         if a_to_power == 1:
                   return True
         for i in xrange(s-1):
                if a_to_power == n - 1:
                          return True
                a_to_power = (a_to_power * a_to_power) % n
        return a_to_power == n - 1

def miller_rabin(n):
	#compute s and d
	d = n - 1
	s = 0
	while d % 2 == 0:
		d >>= 1
		s += 1

	#Run several miller_rabin passes
	for repeat in xrange(20):
		a = randint(2, n-1)
		if not miller_rabin_pass(a, s, d, n):
			return False
	return True

print miller_rabin(n)
\end{verbatim}



\end{document} 


